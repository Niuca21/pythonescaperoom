\documentclass[a4paper 11pt]{article}
    \title{\textbf{Überschrift}} %% Überschrift
    \author{Oliver Oberdick} %% Ersteller
    \date{\today} %% Aktuelles Tagesdatum

%####### für das Literaturverzeichnis
    \usepackage[backend=biber,style=alphabetic,]{biblatex}
    \addbibresource{Studium.bib}
    \bibliography{Studium.bib}

%####### Manipulation der Rändern
%    \addtolength{\topmargin}{-3,5cm}
%    \addtolength{\textwidth}{5cm}
%    \addtolength{\textheight}{8,5cm}
%    \addtolength{\hoffset}{-3,5cm}

\usepackage{fullpage}
\usepackage{graphicx}
\usepackage[german]{babel} % Deutsche Rechtschreibung
\usepackage{pdfpages} % Einfägen von PDF Dateien
\usepackage{struktex} % Struktugramme
\usepackage{makecell}
\usepackage{listings} % Source-Code
\usepackage{xcolor} % Farbvorgaben

%###### Farb definitionen
\definecolor{codegreen}{rgb}{0,0.6,0}
\definecolor{codegray}{rgb}{0.5,0.5,0.5}
\definecolor{codepurple}{rgb}{0.58,0,0.82}
\definecolor{backcolour}{rgb}{0.95,0.95,0.92}

\lstdefinestyle{mystyle}{
    backgroundcolor=\color{backcolour},   
    commentstyle=\color{codegreen},
    keywordstyle=\color{magenta},
    numberstyle=\tiny\color{codegray},
    stringstyle=\color{codepurple},
    basicstyle=\ttfamily\footnotesize,
    breakatwhitespace=false,         
    breaklines=true,                 
    captionpos=b,                    
    keepspaces=true,                 
    numbers=left,                    
    numbersep=5pt,                  
    showspaces=false,                
    showstringspaces=false,
    showtabs=false,                  
    tabsize=2
}

\lstset{style=mystyle}

%####### Bei verwendung einzelner Dateien zur Aufteilung
% \includeonly{chap2} % Nur das aufgeführte Kapitel wird beim erstellen übernommen
% \include{chap2} % einfügen der Datei chap2.tex 

%####### Für die Kapitel Design nach Deutscher Vorgabe
\setlength{\parindent}{0cm} % horizontaler Einzug bei neuen Kapiteln 
\setlength{\parskip}{0,1cm} % vertikaler zusatz Abstand 

\setcounter{tocdepth}{2} % Anzahl / Tiefe der Unterkapitel im Inhaltsverzeichnis 
\setcounter{secnumdepth}{2} % Anzahl / Tiefe der Angezeigten Kapitelnummerierungen

\begin{document}

\maketitle

\begin{abstract}
\begin{center}
Beschreibung zur APL Programmierung 2 der Gruppe HH05
\end{center}
\end{abstract}

\tableofcontents

\thispagestyle{empty}

%##########################################
%##########################################
\newpage
\pagestyle{plain}
\normalsize

%####### Das Deckblatt wird in der Seitenzählung nicht berücksichtigt
\setcounter{page}{1} % Start mit Seite 1 auf dem 2ten Blatt

\section{Story}
% \include{chap2} % einfügen der Datei chap2.tex

Du schaltest deinen Computer ein und siehst nur eine eigenartige Eingabemaske vor dir.
Irgendwer \/ Irgendetwas scheint ihn verändert zu haben.
Jetzt mußt du die Kontrolle zurückerlangen. 'Freundlicherweise' wurden dir einige Hinweise
hinterlassen.
Finde und nutze sie!

\subsection{Levelvorbereitung}

\begin{enumerate}
\addtolength{\itemindent}{0.80cm}
\itemsep0em

\item[1:] erstellen eines zufälligen Verschlüsselungs Key in einer Variable oder speichern in einer Datei 'Dateiname\: Sec\_Key\_HH05.key' zur weiteren nutzung.
\item[2:] evtl könnte auf die Speicherung des Key in einer Datei verzichtet werden, wenn die Eingaben für die Level 1-?? vor dem Level Start generiert werden, dann könnte der Key in einer Variable verbleiben
\item[3:] Verstecken des Key in einer Zufällig ausgewählten Bild-Datei (Eingabedaten für Level 2) Bild-Datei immer wieder neu erstellen und die alte überschreiben.
\item[4:] generieren einer Log-Datei Bsp. wie die ausgabe von \\
netstat -l | grep LISTEN \\
mit zufällig ergänzten Portnummern, welche nicht offen sein sollten.
\item[5:] Nutzen des Key für die Verschlüsselung der Log-Datei auf Bitebene (Eingabedaten für Level 4)
\item[6:] ?

\end{enumerate}

%##########################################
\newpage
\normalsize
\section{Levelbeschreibung}
\subsection{Level 1}
Bsp.: Du schaltest dein Rechner ein und bekommst nur einen eigenartigen Eingabeprompt

\begin{itemize}
\addtolength{\itemindent}{0.80cm}
\itemsep0em
\item ??
\end{itemize}

\subsection{Level 2}
‘Finde das flag / den Key in dem Bild

\begin{itemize}
\addtolength{\itemindent}{0.80cm}
\itemsep0em
\item auf einer normalen Kommandozeile könntest du ihn mit dem Kommando Strings auslesen.
\item merke dir den Key gut.
\item ?
\item nur in dieser Runde lautet der Key ???
\end{itemize}

\subsection{Level 3}
Bsp.: Deine Dateien sind nicht da wo sie sein sollten, suche sie!

\begin{itemize}
\addtolength{\itemindent}{0.80cm}
\itemsep0em
\item ??
\end{itemize}

\subsection{Level 4}
die gefundene Datei muß entschlüsselt werden.

\begin{itemize}
\addtolength{\itemindent}{0.80cm}
\itemsep0em
\item Vielleicht könnte dir der “gemerkte Key” dir behilflich sein.
\item Verschlüsselung mit XOR
\item die Verschlüsselung erfolgte auf Bitebene.
\item wenn der Key nicht lang genug ist, Hilft vielleicht eine aneinanderreihung
\end{itemize}

\subsection{Level 5}
Bsp: Mal schauen was in der Datei zu finden ist

\begin{itemize}
\addtolength{\itemindent}{0.80cm}
\itemsep0em
\item ??
\end{itemize}

\subsection{Level 6}
Bsp: Was sollte jetzt mit den gefundenen Informationen angestellt werden.

\begin{itemize}
\addtolength{\itemindent}{0.80cm}
\itemsep0em
\item ??
\end{itemize}

% \include{chap2} % einfügen der Datei chap2.tex 
%##########################################
%##########################################
\normalsize
\newpage
%####### Einleitung des Anhanges
\appendix{} % Alphabetischen Nummerierung
\section{Programmcode}
\subsection{Vorbereitungen}
\subsubsection{Generell:}
\begin{lstlisting}[language=python, caption=Genereller Escape-Room]
import random
import string
from EscapeRoom import EscapeRoom

import lib.stego as STEGO # Funktionssammlung Oliver Level 2
import lib.crypt as CRYPT # Funktionssammlung Oliver Level 4

class Gruppe_HH_05(EscapeRoom):

    def __init__(self):
        super().__init__()
        self.set_metadata("Veronika, Lucasz & Oliver", __name__)
        self.key = CRYPT.schluessel_erstellen(30) #schluessel erstellen
        self.bild = "static/KEY.jpg"
        STEGO.random_bild(self.bild) # zufaelliges Bild ermitteln und umkopieren
        STEGO.im_bild_verstecken(self.bild , self.key)
        self.verschluesselt = "static/text.crypt"
        CRYPT.schluesselanwendung_datei("static/originale/test.log" ,self.verschluesselt ,self.key )
        
        self.add_level(self.create_level1()) # Veronika
        self.add_level(self.create_level2()) # Oliver
        self.add_level(self.create_level3()) # Veronika
        self.add_level(self.create_level4()) # Oliver
        self.add_level(self.create_level5()) # Lucasz
        self.add_level(self.create_level6()) # Lucasz

    ### LEVELS ###
    # Level 1
    def create_level1(self):
        task_messages = [
            "  <img src=" + self.bild + " alt='The Key you looking for' height='150'/> ",
            "Hi,",
			"das ist zwar kein CTF, aber ein flag ist trotzdem zu suchen",
        ]
        hints = [
            "schau mal im Bild!",
            "suche nach dem flag= ",
            "Eingabedaten sind der Dateiname des Bildes",
            "mit jedem Bild oder neuanfang bekommst du auch eine andere flag",
            "speichern kann nicht schaden, Vorschlag game.key",
            "als encoding wurde 'ISO-8859-1' verwendet",
            "in einem Linux Terminal funktioniert auch der Befehl 'strings [Dateiname]' "
        ]
        return {"task_messages": task_messages, "hints": hints, "solution_function": STEGO.im_bild_finden, "data": self.bild}

    # Level 2
    def create_level2(self):
        task_messages = [
            "  <img src=" + self.bild + " alt='The Key you looking for' height='150'/> ",
            "Hi,",
			"das ist zwar kein CTF, aber ein flag ist trotzdem zu suchen"
        ]
        hints = [
            "schau mal im Bild!",
            "suche nach dem flag= ",
            "Eingabedaten: Dateiname des Bildes",
            "mit jedem Bild oder neuanfang bekommst du auch eine andere flag",
            "speichern kann nicht schaden, Bsp. game.key",
            "als encoding wurde 'ISO-8859-1' verwendet",
            "in einem Linux Terminal funktioniert auch der Befehl 'strings [Dateiname]' "
        ]
        return {"task_messages": task_messages, "hints": hints, "solution_function": STEGO.im_bild_finden, "data": self.bild}

    # Level 3
    def create_level3(self):
        task_messages = [
            "  <img src=" + self.bild + " alt='The Key you looking for' height='400'/> ",
            "Hi,",
			"das ist zwar kein CTF, aber ein flag ist trotzdem zu suchen",
        ]
        hints = [
            "schau mal im Bild!",
            "suche nach dem flag= ",
            "Eingabedaten sind der Dateiname des Bildes",
            "mit jedem Bild oder neuanfang bekommst du auch eine andere flag",
            "speichern kann nicht schaden, Bsp. game.key",
            "als encoding wurde 'ISO-8859-1' verwendet",
            "in einem Linux Terminal funktioniert auch der Befehl 'strings [Dateiname]' "
        ]
        return {"task_messages": task_messages, "hints": hints, "solution_function": STEGO.im_bild_finden, "data": self.bild}

    # Level 4
    def create_level4(self):
        task_messages = [
            "Du hast jetzt einen Dateinamen " + self.verschluesselt + ", schon mar reingeschaut?",
            "zur kontrolle, zeig mir die Zeichen 20 - 70"
        ]
        hints = [
            "kannst du den Inhalt lesen?",
            "Hattest du die flag gespeichert? Bsp. game.key?",
            "Bitweises XOR schon mal gesehen?",
            "als Rueckgabewert die Zeichen 20 - 70 als String zum alsolvieren dieses Level sollten erstmal reichen",
            "Denke drann den Inhalt des Key.File zu nutzen, nicht den Dateinamen",
            "den Key kannst du auch mehrfach hintereinander schreiben, falls er nicht lang genug ist",
            "trotzdem solltest du die komplette Datei bearbeiten und auch wieder speichern. Bsp. ausgabe_encrypt.txt"
        ]
        return {"task_messages": task_messages, "hints": hints, "solution_function": CRYPT.entschluesseln, "data": self.verschluesselt}

    # Level 5
    def create_level5(self):
        task_messages = [
            "  <img src=" + self.bild + " alt='The Key you looking for' height='200'/> ",
            "Hi,",
			"das ist zwar kein CTF, aber ein flag ist trotzdem zu suchen"
        ]
        hints = [
            "schau mal im Bild!",
            "suche nach dem flag= ",
            "Eingabedaten sind der Dateiname des Bildes",
            "mit jedem Bild oder neuanfang bekommst du auch eine andere flag",
            "speichern kann nicht schaden, Vorschlag game.key",
            "als encoding wurde 'ISO-8859-1' verwendet",
            "in einem Linux Terminal funktioniert auch der Befehl 'strings [Dateiname]' "
        ]
        return {"task_messages": task_messages, "hints": hints, "solution_function": STEGO.im_bild_finden, "data": self.bild}

    # Level 6
    def create_level6(self):
        task_messages = [
            "  <img src=" + self.bild + " alt='The Key you looking for' height='200'/> ",
            "Hi,",
			"das ist zwar kein CTF, aber ein flag ist trotzdem zu suchen"
        ]
        hints = [
            "schau mal im Bild!",
            "suche nach dem flag= ",
            "Eingabedaten sind der Dateiname des Bildes",
            "mit jedem Bild oder neuanfang bekommst du auch eine andere flag",
            "speichern kann nicht schaden, Vorschlag game.key",
            "als encoding wurde 'ISO-8859-1' verwendet",
            "in einem Linux Terminal funktioniert auch der Befehl 'strings [Dateiname]' "
        ]
        return {"task_messages": task_messages, "hints": hints, "solution_function": STEGO.im_bild_finden, "data": self.bild}

    ### Hilfsfunktionen ###
    


    ### SOLUTIONS ###


\end{lstlisting}
\flushleft

\subsubsection{Level 1}
\subsubsection{Level 2}
\begin{lstlisting}[language=python, caption=Funktionalitäten für Level 2]
#!/usr/bin/python3

import random
import os

# """ Steganographie
# verstecken und auslesen von Nachrichten in einem Bild.
#
# Oliver Oberdick Matrikel:548933
# """

# Hilfsfunktion nur fuer den EscapeRoom, damit unterschiedliche Bilder genutzt werden
def random_bild(ziel_bild):
	nummer = random.randint(1, 9)
	dst = ziel_bild
	src = "static/originale/Bild_Schluessel_" + str(nummer) + ".jpg"
	if os.name == 'nt':  # pruefen ob Windows
		kopierbefehl = f'copy "{src}" "{dst}"'
	else:
		kopierbefehl = f'cp "{src}" "{dst}"'
	os.system(kopierbefehl)

# Funktion zum Vorbereiten der Level
def im_bild_verstecken(bild_datei, schluessel):
    bild = open(bild_datei, encoding="ISO-8859-1", mode="a+")
    bild.write("flag=" + schluessel)
    bild.close()

# Kontrollfunktion
def im_bild_finden(bild_datei, was="flag="):
    bild = open(bild_datei, encoding="ISO-8859-1", mode="r")
    search = was
    try:
        txt = ""
        byte = bild.read(1)
        while byte != "":
            txt = txt + byte
            byte = bild.read(1)
        pos = txt.find(search) # position des suchstring finden
        pos = pos + len(search) # laenge des suchstrings ueberspringen
        with open("tmp/game.key", 'w') as tmp: # gefundenen Schluessl zwischenspeichern
            tmp.writelines(txt[pos:])
        bild.close()
        return txt[pos:]
    except:
        bild.close()

##
if __name__ == "__main__":
    pass

\end{lstlisting}
\flushleft

\subsubsection{Level 3}
\subsubsection{Level 4}
\begin{lstlisting}[language=python, caption=Funktionalitäten für Level 4]
#!/usr/bin/python3

import random

# """ Verschluesselung 
# Symetrische Verschluesselung mittels XOR auf Bit-Basis
#
# Oliver Oberdick Matrikel:548933
# """

# Hilfsfunktion zum erstellen eines Verschluesselungs key in beliebiger laenge
def schluessel_erstellen(laenge):
        ergebnis = ""
        while len(ergebnis) < laenge:
            zahl = random.randint(48, 122)
            if ((zahl >= 48 and zahl <= 57) or (zahl >= 65 and zahl <= 90) or (zahl >= 97 and zahl <= 122)):
				#  Damit der Schluessel nur aus Zahlen, Grossbuchstaben und Kleinbuchstaben besteht
                ergebnis += chr(zahl)
        return ergebnis

def string_to_binaer(nachricht):
	ergebnis = ""
	for c in nachricht:
		ergebnis += ''.join(format(ord(c), '08b'))
	return ergebnis

def binaer_to_string(nachricht):
	ergebnis = ""
	for i in range(0, len(nachricht), 8):
		ergebnis += chr(int(nachricht[i: i+8], 2))
	return ergebnis

# Ver oder Entschluesseln eines String mittels XOR (Symetrisch)
def schluesselanwendung(was, womit):
	ergebnis = ""
	schluessel = ""
	while len(schluessel) < len(was):
		schluessel += womit
	binaer_schluessel = string_to_binaer(schluessel)
	binaer_nachricht = string_to_binaer(was)
	for i in range(len(binaer_nachricht)):
		ergebnis += str(int(binaer_nachricht[i]) ^ int(binaer_schluessel[i]))
	return binaer_to_string(ergebnis)

# erweiterung, damit auch Dateien ver und entschluesselt werden koennen
def schluesselanwendung_datei(eingabe_datei, ausgabe_datei, schluessel):
	counter = 0  # nur zur kontrolle
	ergebnis = ""  # nur zur kontrolle
	with open(eingabe_datei, 'r') as in_file:
		with open(ausgabe_datei, 'w') as out_file:
			for line in in_file.read():
				counter += 1  # nur zur kontrolle
				tmp = schluesselanwendung(line, schluessel)
				out_file.write(tmp)
				if (counter >= 20 and counter >= 70): # nur zur kontrolle
					ergebnis += tmp # nur zur kontrolle
	return ergebnis # nur zur kontrolle im EscapeRoom Spiel

# Angepasste Funktion, damit zum entschluesseln der Schluessel aus einer Daten genutzt werden kann
def entschluesseln(eingabe, ausgabe="tmp/ausgabe_encrypt.txt", schluessel="tmp/game.key"):
	key = ""
	with open(schluessel, "r") as f:
		key = f.readline()
	return schluesselanwendung_datei(eingabe, ausgabe, key)

##
if __name__ == "__main__":

	key1 = schluessel_erstellen(20)
	key2 = schluessel_erstellen(20)

	print(key1)
	print(key2)

	text = "Hallo du da im Radio!"

	print("Original Text")
	print(text)
	print("Verschluesselt mit Key1")
	text_verschluesselt = schluesselanwendung(text, key1)
	print(text_verschluesselt)
	print("Entschluesselt mit Key1")
	text_entschluesselt = schluesselanwendung(text_verschluesselt, key1)
	print(text_entschluesselt)
	print("entschluesselt mit Key2")
	text_entschluesselt = schluesselanwendung(text_verschluesselt, key2)
	print(text_entschluesselt + "  - mit falschem Key")

	print("-------------------------------------")

	schluesselanwendung_datei("test.txt", "test.crypt", key1)

	schluesselanwendung_datei("test.crypt", "test_entcript.txt", key1)

	print("Dateien fertig")
\end{lstlisting}
\flushleft

\subsubsection{Level 5}
\subsubsection{Level 6}

\subsection{Beispiellösungen}
\subsubsection{Level 1}
\subsubsection{Level 2}
\begin{lstlisting}[language=python, caption=Beispiellösung Aufgabe 2]
# Beispielloesung Level 2

def run(wo, was="flag="):
    bild = open(wo, encoding="ISO-8859-1", mode="r")
    search = was
    try:
        txt = ""
        byte = bild.read(1)
        while byte != "":
            txt = txt + byte
            byte = bild.read(1)
        pos = txt.find(search)
        pos = pos + len(search)
        with open("tmp.txt", 'w') as tmp: # gefundenen Schluessel zwischenspeichern
            tmp.writelines(txt[pos:])
        return txt[pos:]
    except:
        bild.close()
\end{lstlisting}
\flushleft

\subsubsection{Level 3}
\subsubsection{Level 4}
\begin{lstlisting}[language=python, caption=Beispiellösung Aufgabe 4]
# Beispielloesung Level 4

def run(eingabe):
	return schluesselanwendung_datei(eingabe, "ausgabe_encrypt.txt", "tmp.txt")

def string_to_binaer(nachricht):
	ergebnis = ""
	for c in nachricht:
		ergebnis += ''.join(format(ord(c), '08b'))
	return ergebnis

def binaer_to_string(nachricht):
	ergebnis = ""	
	for i in range(0, len(nachricht), 8):
		ergebnis += chr(int(nachricht[i: i+8], 2))
	return ergebnis

def schluesselanwendung(was, womit):
	ergebnis = ""
	schluessel = ""
	while len(schluessel) < len(was):
		schluessel += womit
	binaer_schluessel = string_to_binaer(schluessel)
	binaer_nachricht = string_to_binaer(was)
	for i in range(len(binaer_nachricht)):
		ergebnis += str(int(binaer_nachricht[i]) ^ int(binaer_schluessel[i]))
	return binaer_to_string(ergebnis)

def schluesselanwendung_datei(eingabe_datei, ausgabe_datei, schluessel):
	key = ""
	with open(schluessel, "r") as f:
		key = f.readline()
	counter = 0
	ergebnis = ""
	with open(eingabe_datei, 'r') as in_file:
		with open(ausgabe_datei, 'w') as out_file:
			for line in in_file.read():
				counter += 1
				tmp = schluesselanwendung(line, key)
				out_file.write(tmp)
				if (counter >= 20 and counter >= 70):
					ergebnis += tmp
	return ergebnis
\end{lstlisting}
\flushleft

\subsubsection{Level 5}
\subsubsection{Level 6}

% \include{chap2} % einfügen der Datei chap2.tex 

%####### Einfügen des Literaturverzeichnis
%\printbibliography

%####### Einfügen des Abbildungsverzeichnis
%\listoffigures

%####### Einfügen von PDF Dokumenten
%\includepdf[pages=-, scale=0.8]{Eigenstaendigkeit_Oliver.pdf} % einfügen der Datei chap2.tex 

\end{document}

%##########################################
% Vorlagen / Beispiele
\fbox{\parbox{\linewidth}{}}
% ##
\begin{description}
\addtolength{\itemindent}{0.80cm}
\itemsep0em 

\item[??:] ??
\item ???

\end{description}
% ##
\begin{enumerate}
\addtolength{\itemindent}{0.80cm}
\itemsep0em

\item[??:] ??
\item ??

\end{enumerate}
% ##
\begin{itemize}
\addtolength{\itemindent}{0.80cm}
\itemsep0em

\item[??:] ??
\item ??

\end{itemize}

% Überschriften
\section{Seite 2 ( Section )}
% \include{chap2} % einfügen der Datei chap2.tex
Irgendwelcher Text
\subsection{Subsection}
eine 2te Zeile
\subsubsection{Subsubsection}
Text in der Subsubsection
\paragraph{Paragraph}
Text im Paragraph
\subparagraph{Subparagraph}
Text im Subparagraph
\dots

\addcontentsline{toc}{section}{Eintrag als Zusatzüberschrift ohne Nummer Section}

\addcontentsline{toc}{subsection}{Eintrag als Zusatzüberschrift ohne Nummer SubSection}

Text \ldots \footnote{Irgendeine Fußnote}.

\centering
\begin{struktogramm}(100,50)
\assign{\mbox{}\hfil eingabe von Zahl für n}
	\ifthenelse[15]{2}{2}{ if n größer 0 ? }{true}{false}
		\assign{Summe auf 0 setzen}
		\while{ for i = 1 to n}
		\assign{Summe = Summe + i}
		\whileend
		\assign{Ausgabe Summe}
		\change
		\assign{Ausgabe "Zahl muß positiv sein}
	\ifend
\end{struktogramm} \\


\begin{lstlisting}[language=c++, caption=Programm für Aufgabe 6]
#include <stdio.h>

using namespace std;

int main() {
	int n = 0;
	printf("Bitte geben sie eine Zahl ein: ");
	scanf("%d", &n);
	if(n > 0) {
		// # Berechnung mittels for-Schleife
		int summe_for = 0;
		for(int i = 1; i <= n; i++) {
			summe_for = summe_for + i ;
		}
		printf("Diese Berechnung lief in einer for-Schleife \n");
		printf("Die Summe ist = %d \n", summe_for);
		printf("##\n");
		// # Berechnung mittels while-Schleife
		int summe_while = 0;
		int j = 1;
		while(j <= n) {
			summe_while = summe_while + j ;
			j++ ;
		}
		printf("Diese Berechnung lief in einer while-Schleife \n");
		printf("Die Summe ist = %d \n", summe_while);
		printf("##\n");
		// # Berechnung mittels do-while-Schleife
		int summe_dowhile = 0;
		int k = 1 ;
		do{
			summe_dowhile = summe_dowhile + k ;
			k++ ;
		}while(k <= n);
		printf("Diese Berechnung lief in einer do-while-Schleife \n");
		printf("Die Summe ist = %d \n", summe_dowhile);
		printf("##\n");
	}
	else {
		printf("Die Zahl muss positiv sein");
	}
	return 0;
}
\end{lstlisting}
\flushleft