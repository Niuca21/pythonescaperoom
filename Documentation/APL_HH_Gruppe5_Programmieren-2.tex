\documentclass[a4paper 11pt]{article}
    \title{\textbf{Überschrift}} %% Überschrift
    \author{Veronika\\Tyshchenko\\Matrikel \and Oliver\\Oberdick\\Matrikel \and Lucasz\\Kotula\\Matrikel } %% Ersteller
    \date{\today} %% Aktuelles Tagesdatum

%####### für das Literaturverzeichnis
%    \usepackage[backend=biber,style=alphabetic,]{biblatex}
%    \addbibresource{Studium.bib}
%    \bibliography{Studium.bib}

%####### Manipulation der Rändern
%    \addtolength{\topmargin}{-3,5cm}
%    \addtolength{\textwidth}{5cm}
%    \addtolength{\textheight}{8,5cm}
%    \addtolength{\hoffset}{-3,5cm}

\usepackage{fullpage}
\usepackage{graphicx}
\usepackage[german]{babel} % Deutsche Rechtschreibung
\usepackage{pdfpages} % Einfägen von PDF Dateien
\usepackage{listings} % Einfügen von Source Code
\usepackage{struktex} % Struktugramme
\usepackage{makecell}
\usepackage{listings} % Source-Code
\usepackage{xcolor} % Farbvorgaben

%###### Farb definitionen
\definecolor{codegreen}{rgb}{0,0.6,0}
\definecolor{codegray}{rgb}{0.5,0.5,0.5}
\definecolor{codepurple}{rgb}{0.58,0,0.82}
\definecolor{backcolour}{rgb}{0.95,0.95,0.92}

\lstdefinestyle{mystyle}{
    backgroundcolor=\color{backcolour},   
    commentstyle=\color{codegreen},
    keywordstyle=\color{magenta},
    numberstyle=\tiny\color{codegray},
    stringstyle=\color{codepurple},
    basicstyle=\ttfamily\footnotesize,
    breakatwhitespace=false,         
    breaklines=true,                 
    captionpos=b,                    
    keepspaces=true,                 
    numbers=left,                    
    numbersep=5pt,                  
    showspaces=false,                
    showstringspaces=false,
    showtabs=false,                  
    tabsize=2
}

\lstset{style=mystyle}

%####### Bei verwendung einzelner Dateien zur Aufteilung
% \includeonly{chap2} % Nur das aufgeführte Kapitel wird beim erstellen übernommen
% \include{chap2} % einfügen der Datei chap2.tex 

%####### Für die Kapitel Design nach Deutscher Vorgabe
\setlength{\parindent}{0cm} % horizontaler Einzug bei neuen Kapiteln 
\setlength{\parskip}{0,1cm} % vertikaler zusatz Abstand 

\setcounter{tocdepth}{2} % Anzahl / Tiefe der Unterkapitel im Inhaltsverzeichnis 
\setcounter{secnumdepth}{2} % Anzahl / Tiefe der Angezeigten Kapitelnummerierungen

\begin{document}

\maketitle

\begin{abstract}
\begin{center}
Beschreibung zur APL Programmierung 2 der Gruppe HH05
\end{center}
\end{abstract}

\tableofcontents

\thispagestyle{empty}

%##########################################
%##########################################
\newpage
\pagestyle{plain}
\normalsize

%####### Das Deckblatt wird in der Seitenzählung nicht berücksichtigt
\setcounter{page}{1} % Start mit Seite 1 auf dem 2ten Blatt

\section{Story}

\subsection{Vorgeschichte:}
Dein Rechner wurde von irgendjemandem kompromittiert. \\
Freundlicherweise wurden dir einige hinweise hinterlassen. \\
Folge ihnen um die Kontrolle zurück zu erhalten \\


Du schaltest deinen Computer ein und siehst nur eine eigenartige Eingabemaske vor dir.
Irgendwer \/ Irgendetwas scheint ihn verändert zu haben.
Jetzt mußt du die Kontrolle zurückerlangen. 'Freundlicherweise' wurden dir einige Hinweise
hinterlassen.
Finde und nutze sie!

\subsection{Levelvorbereitung}

\begin{enumerate}
\addtolength{\itemindent}{0.80cm}
\itemsep0em

\item[1:] erstellen eines zufälligen Verschlüsselungs Key in einer Variable oder speichern in einer Datei 'Dateiname\: Sec\_Key\_HH05.key' zur weiteren nutzung.
\item[2:] evtl könnte auf die Speicherung des Key in einer Datei verzichtet werden, wenn die Eingaben für die Level 1-?? vor dem Level Start generiert werden, dann könnte der Key in einer Variable verbleiben
\item[3:] Verstecken des Key in einer Zufällig ausgewählten Bild-Datei (Eingabedaten für Level 2) Bild-Datei immer wieder neu erstellen und die alte überschreiben.
\item[4:] generieren einer Log-Datei Bsp. wie die ausgabe von \\
netstat -l | grep LISTEN \\
mit zufällig ergänzten Portnummern, welche nicht offen sein sollten.
\item[5:] Nutzen des Key für die Verschlüsselung der Log-Datei auf Bitebene (Eingabedaten für Level 4)
\item[6:] ?

\end{enumerate}

%##########################################
\newpage
\normalsize
\section{Levelbeschreibung}
\subsection{Level 1}

Hey Buddy, ich habe jetzt die Kontrolle.\\
Deine Dateien sind verschluesselt.\\
Wenn du dein Passwort wiederhaben willst, folge den Anweisungen.\\
Hier ist mein Wallet: Diese Cockies sind nicht lecker!\\

\begin{itemize}
\addtolength{\itemindent}{0.80cm}
\itemsep0em
\item Schaue die Webseite an und danach stelle fest, sind die Cockies lecker und was die Gangster mit ASCII zu tun haben.
\end{itemize}

\subsection{Level 2}

Your encrypted file wird hier benannt, finde das unten:\\
Nachrichten ansehen

\begin{itemize}
\addtolength{\itemindent}{0.80cm}
\itemsep0em
\item Schreibe deine Lösung so, dass du die Endausgabe Datei liest und die UTC-Zahlen ersetzt.
\end{itemize}


\subsection{Level 3}
Hi,\\
das ist zwar kein CTF, aber ein flag ist trotzdem zu suchen

\begin{itemize}
\addtolength{\itemindent}{0.80cm}
\itemsep0em
\item schau mal im Bild!
\item suche nach dem flag=
\item Eingabedaten sind der Dateiname des Bildes
\item mit jedem Bild oder neuanfang bekommst du auch eine andere flag
\item speichern kann nicht schaden, Bsp. game.key
\item als encoding wurde 'ISO-8859-1' verwendet
\item in einem Linux Terminal funktioniert auch der Befehl 'strings [Dateiname]' v
\end{itemize}


\subsection{Level 4}

Du hast jetzt einen Dateinamen static/text.crypt, schon mar reingeschaut?\\
zur kontrolle, zeig mir die Zeichen 20 - 70

\begin{itemize}
\addtolength{\itemindent}{0.80cm}
\itemsep0em
\item kannst du den Inhalt lesen?
\item Hattest du die flag gespeichert? Bsp. game.key?
\item Bitweises XOR schon mal gesehen?
\item als Rückgabewert die Zeichen 20 - 70 als String zum alsolvieren dieses Level sollten erstmal reichen
\item Denke drann den Inhalt des Key.File zu nutzen, nicht den Dateinamen
\item den Key kannst du auch mehrfach hintereinander schreiben, falls er nicht lang genug ist
\item trotzdem solltest du die komplette Datei bearbeiten und auch wieder speichern. Bsp. ausgabe\_encrypt.txt
\end{itemize}

\subsection{Level 5}

Level 4: Logfile-Analyse\\
Du hast ein Logfile erhalten, das verdächtige Netzwerkaktivitäten enthält.\\
Deine Aufgabe: Extrahiere alle Ports aus dem Logfile und bestimme ihren Status.\\
Achte auf Schlüsselwörter wie secure, attempt, filtered.\\
Lernziele: Textanalyse, Reguläre Ausdrücke, Listen und Dictionaries\\

\begin{itemize}
\addtolength{\itemindent}{0.80cm}
\itemsep0em
\item Nutze re.findall(r"port (\d+)", line), um Portnummern zu extrahieren.
\item Verwende line.lower().strip(), um die Zeile zu normalisieren.
\item Prüfe mit if, ob bestimmte Schlüsselwörter enthalten sind.
\end{itemize}

\subsection{Level 6}

Erstmal nur ein Platzhalter!

\begin{itemize}
\addtolength{\itemindent}{0.80cm}
\itemsep0em
\item ??
\end{itemize}

% \include{chap2} % einfügen der Datei chap2.tex 
%##########################################
%##########################################
\normalsize
\newpage
%####### Einleitung des Anhanges
%\appendix{} % Alphabetischen Nummerierung
\section{Programmcode}
\subsection{Excape Room}
\subsubsection{Der Raum:}

%\lstinputlisting[frame=single,label=beispielcode,caption=Der Raum]{../rooms/Gruppenarbeit_kombiniert.py}. 


\begin{lstlisting}[language=python, caption=der Raum]

import random
import string
from EscapeRoom import EscapeRoom

import time
import re
import lib.stego as STEGO # Funktionssammlung Oliver Level 3
import lib.crypt as CRYPT # Funktionssammlung Oliver Level 4

class Gruppenarbeit_kombiniert(EscapeRoom):

    def __init__(self, response=None):
        super().__init__(response)
        
        self.set_metadata("Veronika, Lucasz & Oliver", __name__)
        
        ## Fuer Level 3-4
        self.key = CRYPT.schluessel_erstellen(30) #schluessel erstellen
        self.bild = "static/KEY.jpg"
        STEGO.random_bild(self.bild) # zufaelliges Bild ermitteln und umkopieren
        STEGO.im_bild_verstecken(self.bild , self.key)
        self.verschluesselt = "static/text.crypt"
        CRYPT.schluesselanwendung_datei("static/originale/test.log" ,self.verschluesselt ,self.key )
        
        ## Fuer Level 5-6
        
        self.add_level(self.create_level1()) # Veronika
        self.add_level(self.create_level2()) # Veronika
        self.add_level(self.create_level3()) # Oliver
        self.add_level(self.create_level4()) # Oliver
        self.add_level(self.create_level5()) # Lucasz
        self.add_level(self.create_level6()) # Lucasz

    ### LEVELS ###
    # Level 1
    def create_level1(self):
        cockie = self.ascii_cockie()
        task_messages = [
            "Hey Buddy, ich habe jetzt die Kontrolle. ",
            "Deine Dateien sind verschluesselt. ",
            "Wenn du dein Passwort wiederhaben willst, folge den Anweisungen.",
            "Hier ist mein Wallet: Diese Cockies sind nicht lecker!"
        ]

        hints = [
            "Schaue die Webseite an und danach stelle fest, sind die Cockies lecker und was die Gangster mit ASCII zu tun haben."
        ]

        self.response.set_cookie("hint", cockie)

        return {
            "task_messages": task_messages,
            "hints": hints,
            "solution_function": self.solution_level1,
            "data": cockie
        }

    # Level 2
    def create_level2(self):
         # Define file paths
        path = "static/template.txt"
        output_path = "static/output.txt"

        # Define placeholders
        self.placeholders = ["{key1}", "{key2}", "{key3}"]

        # Generate decrypted file
        decrypted_path = self.generate_decrypted_file(path, output_path)

        # Count occurrences for internal testing
        solution = self.count_decrypted_words(output_path)
        print("Level 2 solution:", solution)  # z.B."343"

        # Messages for the user
        task_messages = [
            "Your encrypted file wird hier benannt, finde das unten:",
            f"<a href='{decrypted_path}' target='_blank'>Nachrichten ansehen</a>"
        ]

        hints = [
            "Schreibe deine Loesung so, dass du die Endausgabe Datei liest und die UTC-Zahlen ersetzt."
        ]

        return {
            "task_messages": task_messages,
            "hints": hints,
            "solution_function": self.count_decrypted_words,  # This should be your checker
            "data": decrypted_path
        }

    # Level 3
    def create_level3(self):
        task_messages = [
            "  <img src=" + self.bild + " alt='The Key you looking for' height='150'/> ",
            "Hi,",
			"das ist zwar kein CTF, aber ein flag ist trotzdem zu suchen",
        ]
        hints = [
            "schau mal im Bild!",
            "suche nach dem flag= ",
            "Eingabedaten sind der Dateiname des Bildes",
            "mit jedem Bild oder neuanfang bekommst du auch eine andere flag",
            "speichern kann nicht schaden, Bsp. game.key",
            "als encoding wurde 'ISO-8859-1' verwendet",
            "in einem Linux Terminal funktioniert auch der Befehl 'strings [Dateiname]' "
        ]
        return {"task_messages": task_messages, "hints": hints, "solution_function": STEGO.im_bild_finden, "data": self.bild}

    # Level 4
    def create_level4(self):
        task_messages = [
            "Du hast jetzt einen Dateinamen " + self.verschluesselt + ", schon mar reingeschaut?",
            "zur kontrolle, zeig mir die Zeichen 20 - 70"
        ]
        hints = [
            "kannst du den Inhalt lesen?",
            "Hattest du die flag gespeichert? Bsp. game.key?",
            "Bitweises XOR schon mal gesehen?",
            "als Rueckgabewert die Zeichen 20 - 70 als String zum alsolvieren dieses Level sollten erstmal reichen",
            "Denke drann den Inhalt des Key.File zu nutzen, nicht den Dateinamen",
            "den Key kannst du auch mehrfach hintereinander schreiben, falls er nicht lang genug ist",
            "trotzdem solltest du die komplette Datei bearbeiten und auch wieder speichern. Bsp. ausgabe_encrypt.txt"
        ]
        return {"task_messages": task_messages, "hints": hints, "solution_function": CRYPT.entschluesseln, "data": self.verschluesselt}

    # Level 5
    def create_level5(self):
        log_data = """
        Secure connection established on port 443
        Unauthorized access attempt on port 8080
        Port 22 is filtered
        Connection accepted on port 8443
        Unknown activity on port 9999
        """

        parsed_ports = self.parse_logfile(log_data)
#        self.set_solution("malware_ports", parsed_ports)

        task_messages = [
            "<b> Level 4: Logfile-Analyse</b>",
            "Du hast ein Logfile erhalten, das verdaechtige Netzwerkaktivitaeten enthaelt.",
            "Deine Aufgabe: Extrahiere alle Ports aus dem Logfile und bestimme ihren Status.",
            " Achte auf Schluesselwoerter wie <i>secure</i>, <i>attempt</i>, <i>filtered</i>.",
            " Lernziele: Textanalyse, Regulaere Ausdruecke, Listen und Dictionaries"
        ]

        hints = [
            " Nutze <code>re.findall(r\"port (\\d+)\", line)</code>, um Portnummern zu extrahieren.",
            " Verwende <code>line.lower().strip()</code>, um die Zeile zu normalisieren.",
            " Pruefe mit <code>if</code>, ob bestimmte Schluesselwoerter enthalten sind."
        ]

        return {
            "task_messages": task_messages,
            "hints": hints,
            "solution_function": self.check_ports_level4,
            "data": log_data
        }

    # Level 6
    def create_level6(self):
        task_messages = [
            "  <img src=" + self.bild + " alt='The Key you looking for' height='200'/> ",
            "Hi,",
			"das ist zwar kein CTF, aber ein flag ist trotzdem zu suchen"
        ]
        hints = [
            "schau mal im Bild!",
            "suche nach dem flag= ",
            "Eingabedaten sind der Dateiname des Bildes",
            "mit jedem Bild oder neuanfang bekommst du auch eine andere flag",
            "speichern kann nicht schaden, Vorschlag game.key",
            "als encoding wurde 'ISO-8859-1' verwendet",
            "in einem Linux Terminal funktioniert auch der Befehl 'strings [Dateiname]' "
        ]
        return {"task_messages": task_messages, "hints": hints, "solution_function": STEGO.im_bild_finden, "data": self.bild}

    ### Hilfsfunktionen ###
    
        # Level 1. Aufgabe
    def ascii_cockie(self):
        return "67 111 111 107 105 101 109 111 110 115 116 101 114"

        # Level 2. Aufgabe
    def generate_decrypted_file(self, template_path, output_path):
        # Generate UTCs and save as instance variable
        self.utc_list = [
            f"-{self.random_utc_timestamp()}" for _ in self.placeholders]

        with open(template_path, "r", encoding="utf-8") as f:
            text = f.read()

        for ph, utc in zip(self.placeholders, self.utc_list):
            text = text.replace(ph, utc)

        with open(output_path, "w", encoding="utf-8") as f:
            f.write(text)

        return output_path

    @staticmethod
    def random_utc_timestamp(start_year=2000, end_year=2025):
        start = int(time.mktime(time.strptime(
            f"{start_year}-04-12", "%Y-%m-%d")))
        end = int(time.mktime(time.strptime(f"{end_year}-10-31", "%Y-%m-%d")))
        return random.randint(start, end)

		# Level 5. Aufgabe
    def parse_logfile(self, log_text):
        results = []
        lines = log_text.strip().split("\n")

        for line in lines:
            line = line.lower().strip()
            matches = re.findall(r"port (\d+)", line)
            for match in matches:
                port = int(match)
                if "secure" in line or "accepted" in line:
                    status = "open"
                    reason = "secure/accepted"
                elif "attempt" in line or "exposed" in line or "unauthorized" in line:
                    status = "open"
                    reason = "attempt/exposed/unauthorized"
                elif "filtered" in line:
                    status = "closed"
                    reason = "filtered"
                else:
                    status = "closed"
                    reason = "default"

                results.append({
                    "port": port,
                    "status": status,
                    "reason": reason,
                    "raw_line": line
                })

        return results


    ### SOLUTIONS ###

        # Level 1. Loesung
    def solution_level1(self, cockie):
        return "".join(chr(int(n)) for n in cockie.split())

        # Level 2. Loesung
    def count_decrypted_words(self, output_path):
        # Datei lesen
        with open(output_path, "r", encoding="utf-8") as f:
            text = f.read()

        # Alle UTCs im Text finden
        utc_list = re.findall(r"-\d+", text)

        # Vorkommen zaehlen
        counts = {utc: text.count(utc) for utc in utc_list}

        for utc, count in counts.items():
            text = text.replace(utc, f"{count}")

        # Concatenate counts into string like "433"
        name_exe = "".join(str(counts[utc]) for utc in utc_list)
        return name_exe

		# Level 5. Loesung
    def check_ports_level4(self, log_data):
        return self.parse_logfile(log_data)

\end{lstlisting}
\flushleft



\subsubsection{Level 1}

\lstinputlisting[frame=single,label=beispielcode,caption=Ein Beispiel]{../solutions/Platzhalter.py} 

\subsubsection{Level 2}

\lstinputlisting[frame=single,label=beispielcode,caption=Ein Beispiel]{../solutions/Platzhalter.py}

\subsubsection{Level 3}

\lstinputlisting[frame=single,label=beispielcode,caption=Funktionen für Level 3 (Steganographie)]{../rooms/lib/stego.py}. 

\subsubsection{Level 4}

\lstinputlisting[frame=single,label=beispielcode,caption=Funktionen für Level 4 (Symetrische Verschluesselung)]{../rooms/lib/crypt.py}.

\subsubsection{Level 5}

\lstinputlisting[frame=single,label=beispielcode,caption=Ein Beispiel]{../solutions/Platzhalter.py}

\subsubsection{Level 6}

\lstinputlisting[frame=single,label=beispielcode,caption=Ein Beispiel]{../solutions/Platzhalter.py}

\subsection{Beispiellösungen}
\subsubsection{Level 1}

\lstinputlisting[frame=single,label=beispielcode,caption=Beispiellösung Level 1]{../solutions/Level1_Loesung.py}. 

\subsubsection{Level 2}

\lstinputlisting[frame=single,label=beispielcode,caption=Beispiellösung Level 2]{../solutions/Level2_Loesung.py}. 

\subsubsection{Level 3}

\lstinputlisting[frame=single,label=beispielcode,caption=Beispiellösung Level 3]{../solutions/Level3_Loesung.py}. 

\subsubsection{Level 4}

\lstinputlisting[frame=single,label=beispielcode,caption=Beispiellösung Level 4]{../solutions/Level4_Loesung.py}. 

\subsubsection{Level 5}

\lstinputlisting[frame=single,label=beispielcode,caption=Beispiellösung Level 5]{../solutions/Level5_Loesung.py}. 

\subsubsection{Level 6}

\lstinputlisting[frame=single,label=beispielcode,caption=Beispiellösung Level 6]{../solutions/Level6_Loesung.py}. 

% \include{chap2} % einfügen der Datei chap2.tex 

%##########################################
\normalsize
\newpage

\appendix
\section{Eigenständigkeitserklärungen}

%####### Einfügen von PDF Dokumenten
%\includepdf[pages=-, scale=0.8]{Eigenstaendigkeit_Oliver.pdf} % einfügen der Datei chap2.tex 

\end{document}

%####### Einfügen eines Anhanges müßte vor \end{document}
%####### Dabei werden die Sectionsnummern von Zahlen auf Buchstaben umgeschaltet
%\appendix

%####### Einfügen des Literaturverzeichnis müßte vor \end{document}
%\printbibliography

%####### Einfügen des Abbildungsverzeichnis müßte vor \end{document}
%\listoffigures

%##########################################
% Vorlagen / Beispiele
\fbox{\parbox{\linewidth}{}}
% ##
\begin{description}
\addtolength{\itemindent}{0.80cm}
\itemsep0em 

\item[??:] ??
\item ???

\end{description}
% ##
\begin{enumerate}
\addtolength{\itemindent}{0.80cm}
\itemsep0em

\item[??:] ??
\item ??

\end{enumerate}
% ##
\begin{itemize}
\addtolength{\itemindent}{0.80cm}
\itemsep0em

\item[??:] ??
\item ??

\end{itemize}

% Überschriften
\section{Seite 2 ( Section )}
% \include{chap2} % einfügen der Datei chap2.tex
Irgendwelcher Text
\subsection{Subsection}
eine 2te Zeile
\subsubsection{Subsubsection}
Text in der Subsubsection
\paragraph{Paragraph}
Text im Paragraph
\subparagraph{Subparagraph}
Text im Subparagraph
\dots

\addcontentsline{toc}{section}{Eintrag als Zusatzüberschrift ohne Nummer Section}

\addcontentsline{toc}{subsection}{Eintrag als Zusatzüberschrift ohne Nummer SubSection}

Text \ldots \footnote{Irgendeine Fußnote}.

\centering

\begin{struktogramm}(100,50)
\assign{\mbox{}\hfil eingabe von Zahl für n}
	\ifthenelse[15]{2}{2}{ if n größer 0 ? }{true}{false}
		\assign{Summe auf 0 setzen}
		\while{ for i = 1 to n}
		\assign{Summe = Summe + i}
		\whileend
		\assign{Ausgabe Summe}
		\change
		\assign{Ausgabe "Zahl muß positiv sein}
	\ifend
\end{struktogramm} \\


\begin{lstlisting}[language=c++, caption=Hinweis unter dem Programm]
#include <iostream>

using namespace std;

int main() {
	cout << "Hallo du da!" << endl;
	return 0;
}
\end{lstlisting}
\flushleft
